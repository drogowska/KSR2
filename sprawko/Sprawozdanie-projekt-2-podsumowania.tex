\documentclass{classrep}
\usepackage[utf8]{inputenc}
\usepackage{color}
\usepackage{float}
\usepackage{amsmath}
\usepackage{graphicx}
\usepackage{adjustbox}
\usepackage{amsfonts}

\studycycle{Informatyka, studia STACJONARNE, I st.}
\coursesemester{VI}

\coursename{Komputerowe systemy rozpoznawania}
\courseyear{2021/2022}

\courseteacher{prof. dr hab. inż. Adam Niewiadomski}
\coursegroup{poniedziałek, 13:45}

\author{
  \studentinfo{Daria Rogowska}{229989} \and
  \studentinfo{Mateusz Srebnik}{230004} }

\title{Projekt 2.  Podsumowania lingwistyczne relacyjnych baz danych}

\begin{document}
\maketitle

% Opis projektu ma formę artykułu naukowego lub raportu z zadania
% badawczego/doświadczalnego/obliczeniowego (wg indywidualnych potrzeb związanych np. z
% pracą inżynierską/naukową/zawodową). Wzory są numerowane, tablice są numerowane i podpisane nad
% tablicą, rysunki są numerowane i podpisane pod rysunkiem. Podpis rysunku i
% tabeli musi być wyczerpujący (nie ogólnikowy), aby czytelnik nie musiał sięgać do tekstu, aby go zrozumieć.\\
% \indent {\bf Kolejne sekcje sprawozdania są uzupełniane wg wymagań w
% opisie Projektu 2. i Harmonogramie Zajęć na WIKAMP KSR jako efekty zadań w~poszczególnych tygodniach}. 

\section{Cel}

% poprawka
Celem zadania jest stworzenie aplikacji, której główna funkcjonalność odpowiedzialna jest za lingwistyczną agregacje zawartości, wybranego do badania, zbioru danych\cite{database}. 
Program generuje opis w jezyku quasi-naturlanym na podstawie danych liczbowych w zbiorze. 
Podsumowania danych z bazy tworzone są, na podstawie elementów linwistycznego podsumowania tj. kwantyfikatory, kwalifikatory, sumatory oraz podmiot, podanych przez użytkownika, które to stanowią interpretację informacji i wiedzy pozyskanej z dużego zbioru danych.
Wygenerowane podsumowania są prezentowane w formie tekstowej i opiera się na romytych modelach wyrażeń w języku naturlanym.
Część eksperymentalna zadania stanowi określenie 
wpływu wyboru kwantyfikatorów, sumatorów, kwalifikatorów i ich miar jakości na wiarygodność i jakość otrzymanych podsumowań. 


% Zwięzły (2-3 zdania) opis
% problemu, uwzględniający część eksperymentalną i
% implementacyjną.  Opis (własny, nie skopiowany) zawiera przypisy do literatury (bibliografii) zamieszczonej na końcu raportu/sprawozdania
% zgodnie z~Polską Normą (zob. materiały BG PŁ pt. ,,Bibliografia
% załącznikowa'').\\ 
% \indent Opis zawiera minimum teorii ściśle odniesionej do tego konkretnego zadania (zbiór
% danych, liczba zmiennych i rekordów, jakie podsumowania chcesz generować i po
% co, PRZYKŁADY, itp.), tak by inżynier innej specjalności zrozumiał dalszy
% opis tego konkretnego eksperymentu. {\bf Nie przepisuj literatury -- pokaż na
% przykładach jak
% jej elementy wyglądają zastosowane konkretnie do Twojego zadania}.\\
% \noindent {\bf Sekcja uzupełniona jako efekt zadania Tydzień 09 wg Harmonogramu Zajęć na WIKAMP KSR.}


\section{Baza danych, zmienne lingwistyczne, kwantyfikatory lingwistyczne}
% \noindent {\bf Sekcja uzupełniona jako efekt zadania Tydzień 09 wg Harmonogramu Zajęć na WIKAMP KSR.}

\subsection{Charakterystyka podsumowywanej bazy danych}

W celu wykonania projektu wybrano zbiór danych SpeedDating\cite{database}, który zawiera dane zebrane na eksperymentalnych wydarzeniach 'speed dating' na przełomie lat 2002-2004, przeprowadzone przez Columbia Business School. 
Speed dating to rodzaj zorganizowanego randkowania polegający na wyjątkowo krótkich spotkaniach z nieznajomymi. Jeden rekord odpowiada jednemu spotkaniu. Dane opisują osobę randkującą oraz przypisanego jej partnera, oraz wrażenia i wyniki po spotkaniu. 
Zbiór danych może posłużyć np. w celu polepszenia jakości usług agencji matrymonialnych.
Zbiór ten zawiera 8378 rekordów, tego samego typu, a każdy z nich opisany jest na 121 atrybutach, z czego wybrano 11 atrybutów do rozmycia:
\begin{enumerate}
  \item age (oznacza wiek osoby randkującej),
  \item d\_age (oznacza różnice wieku pomiędzy osobą randkującą a jej partnerem podczas spotkania),
  \item importance\_same\_race (oznacza w skali [1-10] ważność posiadania tej samej rasy dla osoby randkującej), 
  \item importance\_same\_religion (oznacza w skali [1-10] ważność posiadania tej samej religii dla osoby randkującej), 
  \item pref\_o\_intelligence (oznacza w skali [0-100] preferowaną inteligencje partnera dla osoby randkującej),
  \item pref\_o\_ambitious (oznacza w skali [0-100] preferowaną ambicje partnera dla osoby randkującej)
  \item tvsports (oznacza w skali [1-10] poziom zainteresowania osoby randkującej oglądaniem sportów w telewizji),
  \item expected\_num\_interested\_in\_me (oznacza oczekiwaną przez osobę randkującą liczbę osób zainteresowanych nią),
  \item guess\_prob\_liked (oznacza w skali [1-10] oczekiwaną szanse na to, że partner polubił osobę randkującą),
  \item funny (oznacza w skali [1-10] jak bardzo zabawna jest osoba randkująca według samej siebie),
  \item sincere (oznacza w skali [1-10] jak bardzo szczera jest osoba randkująca według samej siebie)
\end{enumerate}


Przykładowa wartość rekordu przedstawionego za pomocą wyżej wymienionych atrybutów:
\begin{table}[ht]
\centering
\begin{tabular}{|c|c|}
  \hline
  \textbf{Atrybut} & \textbf{wartość}\\ [0.5ex] 
  \hline
  \hline 
  age & 21 \\ \hline
  d\_age & 6 \\ \hline
  importance\_same\_race & 2 \\ \hline
  importance\_same\_religion & 4 \\ \hline
  pref\_o\_intelligence & 14 \\ \hline
  pref\_o\_ambitious & 50 \\ \hline
  tvsports & 9 \\ \hline
  expected\_num\_interested\_in\_me & 3 \\ \hline
  guess\_prob\_liked & 8 \\ \hline
  funny & 7 \\ \hline
  sincere& 5 \\ \hline
\end{tabular}
\caption{Tabela przedstawiająca wartości wybranych atrybutów przykładowego rekordu zbioru danych SpeedDating \cite{database}}
\end{table}

Wybrane atrybuty przyjmują wartości liczbowe. Ludzie jednak nie posługują się liczbami w sytuacjach gdy istnieje potrzeba opisu pewnych obiektów lub pojęć takich jak na przykład inteligencja, wygląd, szczerość (nie mówimy ,,on jest szczery w stopniu 1 na 10'', tylko ,,on jest fałszywy''). Dlatego też atrybutom wybranym przez nas w tym zadaniu, zostały przypisane zwyczajowe wartości lingwistyczne:
\begin{table}[H]
\centering
\begin{tabular}{|c|c|}
\hline
Atrybut & Zwyczajowe wartości lingwistyczne \\ & nadawane danemu atrybutowi \\ \hline
age & natolatek, młody, \\ & w średnim wieku, w sile wieku, stary \\ \hline
d\_age & brak, niewielka, mała, średnia,\\ &  znacząca, spora \\ \hline
importance\_same\_race & nieistotne, mało istotne, średnio ważne, \\ & znaczące, ważne \\ \hline
importance\_same\_religion & nieistotne, mało istotne, \\ & średnio ważne, znaczące, ważne \\ \hline
pref\_o\_intelligence & głupi, malo inteligentny, przeciętny, \\ & inteligenty, geniusz \\ \hline
pref\_o\_ambitious & nieambitny, średnio ambitny, ambitny \\ \hline
tvsports & niezaintresowany, obojętny, \\ & średnio zainteresowany, \\ & zainteresowany, pasjonat \\ \hline
expected\_num\_interested\_in\_me & brak, niewiele, mało, \\ & kilka, dużo, wiele \\ \hline
guess\_prob\_liked & brak, niewielka, mała, znacząca, spora \\ \hline
funny & nudny, trochę nudny, \\ & przeciętnie zabawny, zabawny \\ \hline
sincere & fałszywy, nieszczery, szczery \\ \hline
\end{tabular}
\caption{Tabela przedstawiająca zwyczajowe wartości lingwistyczne wybranych atrybutów przykładowego rekordu zbioru danych SpeedDating \cite{database}}
\end{table}

Baza danych została zrealizowana w PostgreSQL 14 \cite{postgres}. Poniższy zrzut ekranu przedstawia część bazy danych w programie PgAdmin 4 \cite{pgadmin}.
\begin{figure}[H]
\centering
\includegraphics[scale=0.2]{bazadanych.png}
\caption{Zrzut ekranu programu PgAdmin 4 przedstawiający pierwsze kilka rzędów danych \cite{database}.} 
\end{figure}

%\begin{figure}
%\includegraphics{}
%\caption{}
%\end{figure}

% DONE
% Krótki opis bazy danych wybranej do podsumowywania, źródło, opis treści,
% Liczba rekordów (min. $10\,000$ i koniecznie wszystkie tego
% samego typu), liczba atrybutów możliwych do rozmycia (min. $10$), czyli o stosunkowo dużej
% liczbie możliwych wartości. 
%Zwyczajowe wartości lingwistyczne nadawane wybranym
% atrybutom oraz dlaczego istnieje zwyczaj, zapotrzebowanie/inne powody
% ,,przekałdania'' tych danych na język
% naturalny (a nie formalny) 
% opis użyteczność/zastosowania. 

% DONE ale do sprawdzenia czy bardziej to dodać
% Dodać opis możliwych wartości atrybutów

% TO DO
% Realizacja bazy w wybranym DBMS. Rysunek lub tabela (fragment).

\subsection{Zmienne lingwistyczne (atrybuty/własności obiektów)}

Poniżej zostały przedstawione wzory i wykresy przedstwiające poszczególne zmienne lingwistyczne. \( \mu_z(x) \) oznacza wartość funkcji przynależności zmiennej lingwistycznej \(z\) dla danego atrybutu zbioru danych zależnie od wartości x:

\begin{enumerate}
  \item age

      \begin{equation}
        \mu_{nastolatek} =
          \begin{cases}
            1 & \text{dla } x \in [16,18] \\
            -0.25x +5 & \text{ dla } x \in [18,20]
          \end{cases}  
      \end{equation}

      \begin{equation}
        \mu_{młody} =
        \begin{cases}
            0.2x-3.4 & \text{ dla } x \in [17,22]  \\
            1 & \text{ dla } x \in [23,28] \\
            -0.5x + 15 & \text{ dla } x \in [28,30]
          \end{cases}
      \end{equation}

      \begin{equation}
        \mu_{w\_średnim\_wieku} =
                \begin{cases}
            0.125x-3.375 & \text{ dla } x \in [28,34]  \\
            1 & \text{ dla } x \in [35,40] \\
            -0.25x + 11 & \text{ dla } x \in [40,44]
          \end{cases}
      \end{equation} 

      \begin{equation}
        \mu_{w\_sile\_ wieku} =
        \begin{cases}
            0.2x-8 & \text{ dla } x \in [40,45]  \\
            1 & \text{ dla } x \in [45,49] \\
            -x + 50 & \text{ dla } x \in [19,50]
          \end{cases}
      \end{equation} 

      \begin{equation}
        \mu_{stary} =
        \begin{cases}
            0.2x-9 & \text{ dla } x \in [45,50]  \\
            1 & \text{ dla } x \in [50,65] \\
          \end{cases}
      \end{equation} 

      \begin{figure}[H]
      \includegraphics{fp_age.png}
      \caption{Wykres funkcji przynależności dla atrybutu \(age\).}
      \end{figure}

  \item d\_age 

  \begin{equation}
    \mu_{brak} =
      \begin{cases}
        1 & \text{dla } x =0 \\
        -x +1 & \text{ dla } x \in [0,1]
      \end{cases}  
  \end{equation}

  \begin{equation}
  \mu_{niewielka} =
      \begin{cases}
        0.33x-0.33 & \text{dla } x \in [1,3] \\
        1 & \text{ dla } x=3\\
        -0.5x+3 & \text{dla } x \in [3,5] 
      \end{cases}  
  \end{equation}

  \begin{equation}
    \mu_{mała} =
        \begin{cases}
          0.25x-0.75 & \text{dla } x \in [3,7] \\
          1 & \text{ dla } x=7\\
          -0.33x+3.33 & \text{dla } x \in [7,10] 
        \end{cases}  
    \end{equation}

    \begin{equation}
      \mu_{średnia} =
          \begin{cases}
            0.125x-1.125 & \text{dla } x \in [9,17] \\
            1 & \text{ dla } x\in [17,19]\\
            -0.25x+5.75 & \text{dla } x \in [19,23] 
          \end{cases}  
      \end{equation}

      \begin{equation}
        \mu_{znacząca} =
            \begin{cases}
              0.2x-3 & \text{dla } x \in [15,19] \\
              1 & \text{ dla } x\in [19,22]\\
              -0.1x+3.2 & \text{dla } x \in [22,32] 
            \end{cases}  
        \end{equation}

    \begin{figure}[H]
    \includegraphics{fp_d_age.png}
    \caption{Wykres funkcji przynależności dla atrybutu \(d\_age\).}
    \end{figure}

  \item importance\_same\_race

   \begin{equation}
      \mu_{\text{nieistotne}} =
        \begin{cases}
          1 & \text{dla } x =0 \\
          -x +1 & \text{ dla } x \in [0,1]
        \end{cases}  
    \end{equation}

    \begin{equation}
      \mu_{\text{mało istotne}} =
        \begin{cases}
          x-1 & \text{dla } x \in [1,2] \\
          1 & \text{ dla } x =2 \\
          -x+3 & \text{ dla } x \in [2,3]       
        \end{cases}  
    \end{equation}

    \begin{equation}
      \mu_{\text{średnio ważne}} =
        \begin{cases}
          x-2 & \text{dla } x \in [2,3] \\
          1 & \text{ dla } x \in [3,4] \\
          -x+5 & \text{ dla } x \in [4,5]       
        \end{cases}  
    \end{equation}

    \begin{equation}
      \mu_{\text{przeciętne}} =
        \begin{cases}
          x-4 & \text{dla } x \in [4,5] \\
          1 & \text{ dla } x =5 \\
          -x+6 & \text{ dla } x \in [5,6]       
        \end{cases}  
    \end{equation}

    \begin{equation}
      \mu_{\text{znaczące}} = e^{-{(x-6)}^2}, \text{ dla } x \in [5,7]
    \end{equation}

    \begin{equation}
      \mu_{\text{ważne}} = e^{-(\frac{x-9}{1.5})^2}, \text{ dla } x \in [7,10]
    \end{equation}

      \begin{figure}[H]
      \includegraphics{fp_israce.png}
      \caption{Wykres funkcji przynależności dla atrybutu \(importance\_same\_race\).}
      \end{figure}

  

  \item importance\_same\_religion 
    \begin{equation}
      \mu_{\text{nieistotne}} =
        \begin{cases}
          1 & \text{dla } x =0 \\
          -x +1 & \text{ dla } x \in [0,1]
        \end{cases}  
    \end{equation}

    \begin{equation}
      \mu_{\text{mało istotne}} =
        \begin{cases}
          x-1 & \text{dla } x \in [1,2] \\
          1 & \text{ dla } x =2 \\
          -x+3 & \text{ dla } x \in [2,3]       
        \end{cases}  
    \end{equation}

    \begin{equation}
      \mu_{\text{średnio ważne}} =
        \begin{cases}
          x-2 & \text{dla } x \in [2,3] \\
          1 & \text{ dla } x \in [3,4] \\
          -x+5 & \text{ dla } x \in [4,5]       
        \end{cases}  
    \end{equation}

    \begin{equation}
      \mu_{\text{przeciętne}} =
        \begin{cases}
          x-4 & \text{dla } x \in [4,5] \\
          1 & \text{ dla } x =5 \\
          -x+6 & \text{ dla } x \in [5,6]       
        \end{cases}  
    \end{equation}

    \begin{equation}
      \mu_{\text{znaczące}} = e^{-{(x-6)}^2}, \text{ dla } x \in [5,7]
    \end{equation}

    \begin{equation}
      \mu_{\text{ważne}} = e^{-(\frac{x-9}{1.5})^2}, \text{ dla } x \in [7,10]
    \end{equation}

      \begin{figure}[H]
      \includegraphics{fp_isrel.png}
      \caption{Wykres funkcji przynależności dla atrybutu \(importance\_same\_religion\).}
      \end{figure}
  

  \item pref\_o\_intelligence
  \begin{equation}
    \mu_{głupi} =
      \begin{cases}
        1 & \text{ dla } x \in [0 ,1] \\
        -0.5x+1.5 & \text{ dla } x \in [1,3]       
      \end{cases}  
  \end{equation}

  \begin{equation}
    \mu_{mało\_inteligentny} =
      \begin{cases}
        0.33x-0.66 & \text{ dla } x \in [2,5]\\
        1 & \text{ dla } x \in [5,12] \\
        -0.25x+4 & \text{ dla } x \in [12,16]       
      \end{cases}  
  \end{equation}

  \begin{equation}
    \mu_{przeciętny} =
      \begin{cases}
        0.166x-2.33 & \text{ dla } x \in [14,20]\\
        1 & \text{ dla } x \in [20,30] \\
        -0.2x+7 & \text{ dla } x \in [30,35]       
      \end{cases}  
  \end{equation}

  \begin{equation}
    \mu_{inteligentny} =
      \begin{cases}
        0.083x-2.67 & \text{ dla } x \in [32,44]\\
        1 & \text{ dla } x \in [44,58] \\
        -0.16x+10.3 & \text{ dla } x \in [58,64]       
      \end{cases}  
  \end{equation}

  \begin{equation}
    \mu_{geniusz} =
      \begin{cases}
        0.2x-11.4 & \text{ dla } x \in [57,62] \\
        1 & \text{ dla } x \in [62,67] \\   
      \end{cases}  
  \end{equation}
  
  \begin{figure}[H]
    \includegraphics{fp_poi.png}
    \caption{Wykres funkcji przynależności dla atrybutu \(pref\_o\_intelligence\).}
    \end{figure}

  \item pref\_o\_ambitious
  \begin{equation}
    \mu_{nieambitny} = e^{-(\frac{x-6}{8})^2}, \text{ dla } x \in [0,24]
  \end{equation}

  \begin{equation}
    \mu_{średnio\_ambitny} = e^{-(\frac{x-30}{12})^2}, \text{ dla } x \in [5,54]
  \end{equation}

  \begin{equation}
    \mu_{ambitny} = e^{-(\frac{x-45}{5})^2}, \text{ dla } x \in [30,55]
  \end{equation}
  
  \begin{figure}[H]
    \includegraphics{fp_poa.png}
    \caption{Wykres funkcji przynależności dla atrybutu \(pref\_o\_ambitious\).}
    \end{figure}
  
  \item tvsports 
  \begin{equation}
    \mu_{niezaintresowany} =
      \begin{cases}
        1 & \text{ dla } x \in [0,2] \\
        -0.33x+1.67 & \text{ dla } x \in [2,5]       
      \end{cases}  
  \end{equation}

  \begin{equation}
    \mu_{obojętny} =
      \begin{cases}
        0.5x-1 & \text{ dla } x \in [2,4]\\
        1 & \text{ dla } x=4 \\
        -0.33x+2.34 & \text{ dla } x \in [4,7]       
      \end{cases}  
  \end{equation}

  \begin{equation}
    \mu_{średnio\_zainteresowany} =
      \begin{cases}
        0.5x-2 & \text{ dla } x \in [4,6]\\
        1 & \text{ dla } x \in [6,8] \\
        -x+9 & \text{ dla } x \in [8,9]       
      \end{cases}  
  \end{equation}

  \begin{equation}
    \mu_{zainteresowany} =
      \begin{cases}
        x-8 & \text{ dla } x \in [8,9]\\
        1 & \text{ dla } x =9 \\
        -0.5x+5.5 & \text{ dla } x \in [9,11]       
      \end{cases}  
  \end{equation}

  \begin{equation}
    \mu_{pasjonat} =
      \begin{cases}
        0.5x-4.5 & \text{ dla } x \in [9,11]\\
        1 & \text{ dla } x \in [11,12] \\
     
      \end{cases}  
  \end{equation}

  \begin{figure}[H]
    \includegraphics{fp_tvs.png}
    \caption{Wykres funkcji przynależności dla atrybutu \(tvsports\).}
    \end{figure}

  \item expected\_num\_interested\_in\_me
  
  \begin{equation}
    \mu_{brak} =
      \begin{cases}
        1 & \text{ dla } x \in [0,1] \\
        0.5x+1.5  & \text{ dla } x \in [1,3]
      \end{cases}  
  \end{equation} 

  \begin{equation}
    \mu_{niewiele} =
      \begin{cases}
        0.5x-0.5 & \text{ dla } x \in [1,3] \\
        1 & \text{ dla } x \in [3,4] \\
        -x+5  & \text{ dla } x \in [4,6]
      \end{cases}  
  \end{equation}
  
  \begin{equation}
    \mu_{mało} =
      \begin{cases}
        x-4 & \text{ dla } x \in [4,5] \\
        1 & \text{ dla } x =5 \\
        0.5x+3.5  & \text{ dla } x \in [5,7]
      \end{cases}  
  \end{equation}

  \begin{equation}
    \mu_{kilka} =
      \begin{cases}
        x-6 & \text{ dla } x \in [6,7] \\
        1 & \text{ dla } x =7 \\
        -x+4  & \text{ dla } x \in [7,8]
      \end{cases}  
  \end{equation}
  
  \begin{equation}
    \mu_{dużo} =
      \begin{cases}
        x+8 & \text{ dla } x \in [7,9] \\
        1 & \text{ dla } x =9 \\
        -x+10  & \text{ dla } x \in [9,11]
      \end{cases}  
  \end{equation}

  \begin{equation}
    \mu_{wiele} =
      \begin{cases}
        0.5x-4.5 & \text{ dla } x \in [8,11] \\
        1 & \text{ dla } x \in [11,12]
      \end{cases}  
  \end{equation}

  \begin{figure}[H]
    \includegraphics{fp_enim.png}
    \caption{Wykres funkcji przynależności dla atrybutu \(expected\_num\_interested\_in\_me\).}
    \end{figure}

  \item guess\_prob\_liked
  \begin{equation}
    \mu_{brak} =
      \begin{cases}
        1 & \text{ dla } x \in [0,1] \\
        -x+2  & \text{ dla } x \in [1,2]
      \end{cases}  
  \end{equation}

  \begin{equation}
    \mu_{niewielka} =
      \begin{cases}
        0.5x-0.5 & \text{ dla } x \in [1,3] \\
        1 & \text{ dla } x =3 \\
        -x+4  & \text{ dla } x \in [3,4]
      \end{cases}  
  \end{equation}

  \begin{equation}
    \mu_{mała} =
      \begin{cases}
        0.5x-1.5 & \text{ dla } x \in [3,5] \\
        1 & \text{ dla } x =5 \\
        -x+6  & \text{ dla } x \in [5,6]
      \end{cases}  
  \end{equation}

  \begin{equation}
    \mu_{znaczaca} =
      \begin{cases}
        0.34x-1.67 & \text{ dla } x \in [5,8] \\
        1 & \text{ dla } x \in [8,9] \\
        -x+9  & \text{ dla } x \in [9,10]
      \end{cases}  
  \end{equation}

  \begin{equation}
    \mu_{spora} =
      \begin{cases}
        0.5x-4.5 & \text{ dla } x \in [9,11] \\
        1 & \text{ dla } x \in [11,12] \\
      \end{cases}  
  \end{equation}

  \begin{figure}[H]
    \includegraphics{fp_gpl.png}
    \caption{Wykres funkcji przynależności dla atrybutu \(guess\_prob\_liked \).}
    \end{figure}

  \item funny
  \begin{equation}
    \mu_{nudny} =
      \begin{cases}
        1 & \text{ dla } x \in [0,2] \\
        -0.5x-1.5 & \text{ dla } x \in [2,3] 
      \end{cases}  
  \end{equation}
 
  \begin{equation}
    \mu_{trochę\_nudny} =
      \begin{cases}
        0.33x-0.34 & \text{ dla } x \in [1,4] \\
        1 & \text{ dla } x \in [4,5] \\
        -x+6 & \text{ dla } x \in [5,6] 
      \end{cases}  
  \end{equation}
  
  \begin{equation}
    \mu_{przeciętnie\_zabawny} =
      \begin{cases}
        x-5 & \text{ dla } x \in [5,6] \\
        1 & \text{ dla } x \in [6,7] \\
        -0.34x-3.34 & \text{ dla } x \in [7,10] 
      \end{cases}  
  \end{equation}

  \begin{equation}
    \mu_{zabawny} =
      \begin{cases}
        x-9 & \text{ dla } x \in [9,10] \\
        1 & \text{ dla } x \in [10,13] \\
        
      \end{cases}  
  \end{equation}

  \begin{figure}[H]
    \includegraphics{fp_f.png}
    \caption{Wykres funkcji przynależności dla atrybutu \(funny\).}
    \end{figure}

  \item sincere
  
  \begin{equation}
    \mu_{fałszywy} =
      \begin{cases}
        1 & \text{ dla } x \in [0,2] \\
        -0.34x+2 & \text{ dla } x \in [2,6] \\
      \end{cases}  
  \end{equation}

  \begin{equation}
    \mu_{nieszczery} =  e^{-(\frac{x-7}{2})^2}, dla x \in [4,11]
  \end{equation}

  \begin{equation}
    \mu_{szczery} =
      \begin{cases}
        0.34x-2.67 & \text{ dla } x \in [8,11] \\
        1 & \text{ dla } x \in [11,13] \\
      \end{cases}  
  \end{equation}

  
  \begin{figure}[H]
    \includegraphics{fp_s.png}
    \caption{Wykres funkcji przynależności dla atrybutu \(sincere\).}
    \end{figure}

\end{enumerate}


%Zmienne lingwistyczne dla wybranych 10 atrybutów z bazy danych, przedstawione w
%formie wykresów funkcji przynależności i wzorów analitycznych, wymienione etykiety oraz objaśnione wszystkie
%symbole ułatwiające czytelnikowi ich zrozumienie \cite{zadrozny06}. {\bf Zbędne jest
%cytowanie definicji}. Konieczne {\bf precyzyjnie podane przestrzenie rozważań każdej
%zmiennej lingwistycznej}, wzory i wykresy dla każdej wartości/etykiety.

\subsection{Kwantyfikatory lingwistyczne (liczności obiektów)}

Poniżej zostały przedstawione wzory i wykresy przedstwiające poszczególne kwantyfikatory lingwistyczne, zarówno względne jak i absolutne. 
\( \mu_z(x) \) oznacza wartość funkcji przynależności kwantyfikatora lingwistycznego \(z\) dla danego atrybutu zbioru danych zależnie od wartości x:

 \begin{equation}
    \mu_{\text{Prawie żadne}} = 
    \begin{cases}
        1 & \text{ dla } x \in [0,50] \\
        2 - 0.02x & \text{ dla } x \in [50,100] \\
      \end{cases}
  \end{equation}

  \begin{equation}
    \mu_{\text{Trochę}} =
      \begin{cases}
        0.005x & \text{ dla } x \in [0,200] \\
        1 & \text{ dla } x \in [200,1500] \\
        -0.005x & \text{ dla } x \in [1500,1700] \\
      \end{cases}  
  \end{equation}
  
  \begin{equation}
    \mu_{\text{ok. 1/3}} = e^{-(\frac{x-2700}{800})^2}, \text{ dla } x \in [1000,5000]
  \end{equation}

   \begin{equation}
    \mu_{\text{Mniej więcej połowa}} =
      \begin{cases}
        0.0025x - 9.4725 & \text{ dla } x \in [3789,4189] \\
        -0.0025x + 11.4725 & \text{ dla } x \in [4189,4589] \\
      \end{cases}  
  \end{equation}
  
  \begin{equation}
    \mu_{\text{Większość}} =
      \begin{cases}
        0.002x - 8.2 & \text{ dla } x \in [4100,4600] \\
        1 & \text{ dla } x \in [4600,8378] \\
      \end{cases}  
  \end{equation}
  
  \begin{equation}
    \mu_{\text{Zdecydowana większość}} =
      \begin{cases}
        0.001x - 7.3 & \text{ dla } x \in [7300,8300] \\
        1 & \text{ dla } x \in [8300,8378] \\
      \end{cases}  
  \end{equation}
  
 \begin{figure}[H]
 \centering
   \includegraphics{fp_qr.png}
   \caption{Wykres funkcji przynależności dla kwantyfikatorów względnych użytych w zadaniu}
 \end{figure}

  
 \begin{equation}
  \mu_{\text{mniej niż 1000}} =
    \begin{cases}
      1 & \text{ dla } x \in [0,600] \\
      0.0034x - 3.34 & \text{ dla } x \in [600,1000] \\
    \end{cases}  
\end{equation}

\begin{equation}
  \mu_{\text{ponad 1000}} = e^{-(\frac{x-1500}{500})^2}, \text{ dla } x \in [300,2700]
\end{equation}

\begin{equation}
  \mu_{\text{ok. 3000}} =
    \begin{cases}
      0.001x - 1.6 & \text{ dla } x \in [1600,2600] \\
      1 & \text{ dla } x \in [2600,4000] \\
      0.001x +5 & \text{ dla } x \in [4000,5000] \\
    \end{cases}  
\end{equation}

\begin{equation}
  \mu_{\text{ponad 5000}} =e^{-(\frac{x-5000}{1000})^2}, \text{ dla } x \in [2800,6900]
\end{equation}

\begin{equation}
  \mu_{\text{ok. 8000}} =
    \begin{cases}
      0.0004x - 2.2 & \text{ dla } x \in [5500,8000] \\
      1 & \text{ dla } x \in [8000,8400] \\
    \end{cases}  
\end{equation}

 \begin{figure}[H]
  \includegraphics{fp_qa.png}
  \caption{Wykres funkcji przynależności dla kwantyfikatorów absolutnych użytych w zadaniu}
\end{figure}

%Jw. kwantyfikatory lingwistyczne -- opisane etykietami, wykresami funkcji
%przynależności i wzorami analitycznymi. Uzasadnione wiedzą dziedzinową  
%{\bf zakresy i etykiety}. Precyzyjnie podane przestrzenie rozważań każdego kwantyfikatora 
%lingwistycznego/rozmytego, wzory i wykresy dla każdej wartości/etykiety. Opisy własne z~przypisami do literatury, tak by inżynier innej specjalności zrozumiał dalszy
%opis tego konkretnego ćwiczenia/eksperymentu.  


\section{Narzędzia obliczeniowe: wybór/implementacja. Diagram UML pakietu
obliczeń rozmytych i~generatora podsumowań. Instrukcja użytkownika}

Aplikacja została napisana w technologii Java w wersji 17, z wykorzystaniem biblioteki JavaFX \cite{javafx} w celu stworzenia prostego
okienkowego interfejsu użytkownika. W celu odczytania danych zawarytch w bazie danych \cite{database} użyto Spring Framework \cite{spring}. \\

Implementacja została podzielona na dwa moduły warstwy logiki i widoku, wykorzystany został wzorzec projektowy Model View Controler. Z kolei cześć odpowiedzialna za logikę aplikacji składa się z poszczególnych pakietów:
\begin{enumerate}
  \item functions - pakiet reprezentujący funkcje przynależności (klasa abstrakcyjna MembershipFunction i rozsrzerzające ją klasy). 
  \begin{figure}[H]
    \centering
    \includegraphics[scale = 0.2]{fun}
    \caption{Diagram UML pakietu \(functions\)}
  \end{figure}
  \item fuzzy - pakiet reprezentujący poszczególne elementy podsumowania, jak i samo podsumowanie lingwistyczne oraz metryki użyte to obliczeń jakości wykonywanych operacji. Zawiera międzyinnymi klasy: LinguisticVariable reprezentującą zmienną lingwistyczną, LinguisticSummary - podsumowanie lingwistyczne.
  \begin{figure}[H]    
    \centering
    \includegraphics[scale = 0.16]{fuzzy}
    \caption{Diagram UML pakietu \(fuzzy\)}
  \end{figure}
  \item set - pakiet reprezentujący obiekt zbioru, zarówno rozmumianego w sensie klasycznym jak i rozmytym. Zawiera klase abstrakcyjną Set, która rozszerza klasę bibliioteczną ArrayList oraz rozszerzające ją klasę FuzzySet.
  \begin{figure}[H]
    \centering
    \includegraphics[scale = 0.2]{set}
    \caption{Diagram UML pakietu \(set\)}
  \end{figure}
  \item data - pakiet reprezentujący pojedyńczy rekord odczytany z bazy danych.
   \begin{figure}[H]
    \centering
    \includegraphics[scale = 0.2]{data}
    \caption{Diagram UML pakietu \(data\)}
  \end{figure}
\end{enumerate}



  \begin{figure}[H]
    \includegraphics[scale = 0.08]{all}
    \caption{Diagram UML modułu logiki}
  \end{figure}
  Poniżej zamieszczono poszczególne widoki GUI aplikacji:
  
  \begin{figure}[H]
    \centering
    \includegraphics[scale = 0.5]{gui1}
    \caption{Widok GUI aplikacji dla użytkownika niezaawansowanego pozwalający na generowanie podsumowań poprzez wybranie między innymi kwantyfikatora, podmiotu oraz określenie wag dla poszczególnych metryk}
  \end{figure}
  
  \begin{figure}[H]
    \centering
    \includegraphics[scale = 0.5]{gui2}
    \caption{Widok GUI aplikacji dla użytkownika zaawansowanego pozwalający na definiowanie własnych etykiet i funckji przynależności dla nowych kwantyfikatorów, sumaryzatorów i kwalifikatorów}
  \end{figure}
  
  \begin{figure}[H]
    \centering
    \includegraphics[scale = 0.5]{gui3}
    \caption{Widok GUI aplikacji przedstawiający wyniki podsumowań}
  \end{figure}


%\noindent {\bf Sekcja uzupełniona jako efekt zadania Tydzień 10 wg Harmonogramu Zajęć na WIKAMP KSR.}
%Diagram UML i zwięzły opis pakietu obliczeń rozmytych: źródło pakietu
%(zewnętrzny/własny/hybrydowy), przypis do literatury/źródeł.

% Krótka charakterystyka
%najważniejszych klas i podstawowych dla zadania ich metod. \\

%Diagram UML generatora podsumowań (warstwy obliczeniowej oraz interfejsu
%użytkownika). Krótki ilustrowany opis jak użytkownik może korzystać z aplikacji, w~szczególności
%wprowadzać parametry  podsumowań, odczytywać wyniki oraz definiować własne etykiety i
%kwantyfikatory.\\

%Wersja JRE i inne wymogi niezbędne do uruchomienia aplikacji przez użytkownika na własnym komputerze. 


\section{ Jednopodmiotowe podsumowania lingwistyczne. Miary jakości, podsumowanie optymalne}
Wyniki kolejnych eksperymentów wg punktów 2.-4. opisu projektu 2.  Listy podsumowań
jednopodmiotowych i tabele/rankingi podsumowań dla danych atrybutów obowiązkowe i dokładnie opisane w ,,captions'' (tytułach), konieczny opis kolumn i wierszy tabel. Dla każdego podsumowania podane miary jakości oraz miara jakości podsumowania
optymalnego. {\bf Wzorów podsumowań ani miar nie należy przepisywać ani cytować, wystarczy podać literaturę, ale
należy skomentować co oznaczają i jaką informacje niosą wybrane miary w wybranych
przypadkach.}\\
\noindent {\bf Sekcja uzupełniona jako efekt zadania Tydzień 11 wg Harmonogramu Zajęć na WIKAMP KSR.}

\section{Wielopodmiotowe podsumowania lingwistyczne i~ich miary jakości} 
Wyniki kolejnych eksperymentów wg punktów 2.-4. opisu projektu 2. Uzasadnienie i
metoda podziału zbioru danych na rozłączne podmioty. Listy podsumowań
wielopodmiotowych i tabele/rankingi podsumowań dla danych atrybutów obowiązkowe i
dokładnie opisane w ,,captions'' (tytułach), konieczny opis kolumn i wierszy tabel.
{\bf Wzorów podsumowań ani miar nie należy przepisywać ani cytować, wystarczy podać literaturę, ale
należy skomentować co oznaczają i jaką informacje niosą wybrane miary w wybranych
przypadkach.}Konieczne uwzględnienie wszystkich 4-ch form podsumowań wielopodmiotowych. 
\\ 

** Możliwe sformułowanie zagadnienia wielopodmiotowego podsumowania optymalnego **.\\
\indent {** Ewentualne wyniki realizacji punktu ,,na ocenę 5.0'' wg opisu Projektu 2. i ich porównanie do wyników z
części obowiązkowej **.}\\

\noindent {\bf Sekcja uzupełniona jako efekt zadania Tydzień 12 wg Harmonogramu Zajęć
na WIKAMP KSR.}


\section{Dyskusja, wnioski}
Dokładne interpretacje uzyskanych wyników w zależności od parametrów klasyfikacji
opisanych w punktach 3.-4 opisu Projektu 2. 
Szczególnie istotne są wnioski o charakterze uniwersalnym, istotne dla podobnych zadań. 
Omówić i wyjaśnić napotkane problemy (jeśli były). Każdy wniosek/problem powinien mieć poparcie
w przeprowadzonych eksperymentach (odwołania do konkretnych wyników: tabel i miar
jakości). Ocena które wybrane kwantyfikatory, sumaryzatory, kwalifikatory i/lub ich
miary jakości mają małe albo duże znaczenie dla wiarygodności i jakości otrzymanych
agregacji/podsumowań.  \\
\underline{Dla końcowej oceny jest to najważniejsza sekcja} sprawozdania, gdyż prezentuje poziom
zrozumienia rozwiązywanego problemu.\\

** Możliwości kontynuacji prac w obszarze logiki rozmytej i wnioskowania rozmytego, zwłaszcza w kontekście pracy inżynierskiej,
magisterskiej, naukowej, itp. **\\

\noindent {\bf Sekcja uzupełniona jako efekt zadań Tydzień 11 i Tydzień 12 wg
Harmonogramu Zajęć na WIKAMP KSR.}


\section{Braki w realizacji projektu 2.}
Wymienić wg opisu Projektu 2. wszystkie niezrealizowane obowiązkowe elementy projektu, ewentualnie
podać merytoryczne (ale nie czasowe) przyczyny tych braków. 


\begin{thebibliography}{99}
 \bibitem{niewiadomski19} A. Niewiadomski, Zbiory rozmyte typu 2. Zastosowania w reprezentowaniu informacji.  Seria „Problemy współczesnej informatyki” pod redakcją L. Rutkowskiego. Akademicka Oficyna Wydawnicza EXIT, Warszawa, 2019.
\bibitem{zadrozny06} S. Zadrożny, Zapytania nieprecyzyjne i lingwistyczne podsumowania baz danych, EXIT, 2006, Warszawa
\bibitem{niewiadomski08} A. Niewiadomski, Methods for the Linguistic Summarization of Data: Applications of Fuzzy Sets and Their Extensions, Akademicka Oficyna Wydawnicza EXIT, Warszawa, 2008.
\bibitem{database} SpeedDating Dataset, \url{https://www.openml.org/search?type=data&sort=runs&status=active&id=40536}
\bibitem{postgres} strona internetowa PostgreSQL, \url{https://www.postgresql.org}
\bibitem{pgadmin} strona internetowa PgAdmin 4, \url{https://www.pgadmin.org}
\bibitem{javafx} JavaFX, \url{https://openjfx.io}
\bibitem{spring} Spring Framework, \url{https://spring.io}

\end{thebibliography}

Literatura zawiera wyłącznie źródła recenzowane i/lub o potwierdzonej wiarygodności,
możliwe do weryfikacji i cytowane w sprawozdaniu. 
\end{document}
